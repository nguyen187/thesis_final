
\documentclass[12pt,a4paper,oneside]{book} % twoside for draf

%\usepackage{babel}
\usepackage[utf8]{vietnam}
% \usepackage[utf8]{inputenc}
\usepackage{tipa}
\usepackage{amssymb}
\usepackage{graphicx}
\usepackage{subcaption}
\usepackage{booktabs}
\usepackage{mathptmx}	% same Time New Roma
\usepackage{amsmath}
\usepackage{amssymb}
\usepackage{amsfonts}
\usepackage{sty/ipa}
%\renewcommand{\rmdefault}{phv} % Arial
%\renewcommand{\sfdefault}{phv} % Arial
\usepackage{array}
\newcolumntype{P}[1]{>{\centering\arraybackslash}p{#1}}
\newcolumntype{M}[1]{>{\centering\arraybackslash}m{#1}}
\usepackage{fancyhdr}
\usepackage{multirow}
\usepackage{algorithm2e}
\usepackage{hyperref}
\usepackage{float}

\usepackage{sty/hcmusthesis}

\usepackage{graphicx}
\graphicspath{ {image/} }
\setcounter{secnumdepth}{2}
\crname{LUẬN VĂN TỐT NGHIỆP ĐẠI HỌC}
\ctname{Development of Yolo machine learning \\model and real-time streaming operational parameters of CO2 micro algae capture pilot}
\cstuname{
ĐÀO THANH NGUYÊN-20280068
}

% \csCouncil{KHOA TOÁN-TIN}
\csReviewer{TS. TRỊNH NGỌC TRUNG}
\csSupervise{TS. HOÀNG VĂN HÀ}

\cttime{7/2024}

\thesislayout

\begin{document}
%-	Bìa cứng - màu xanh dương, chữ mạ vàng (xem mẫu đính kèm)
%-	Trang tên (tờ lót): chất liệu giấy, nội dung giống như bìa LV
%-	Ở gáy LV: in nhan đề LV (có thể in tóm tắt nếu nhan đề quá dài), size 15 – 17
%-	Phiếu Nhiệm vụ LV, chấm điểm Hướng dẫn & Phản biện (đã ký): nhận từ GVHD & GVPB sau khi bảo vệ (theo lịch hẹn).
%-	Lời cam đoan
%-	Lời cảm ơn/ Lời ngỏ
%-	Tóm tắt LV
%-	Mục lục
%-	Danh mục, bảng biểu, hình ảnh, ... (nếu có)
%-	Nội dung LV
%-	Danh mục TL tham khảo
%-	Phụ lục (nếu có)

\coverpage

\frontmatter


\begin{declaration}
	...
\end{declaration}

\begin{acknowledgments}

	Để hoàn thành kì đề cương luận văn này, tôi tỏ lòng biết ơn sâu sắc đến tiến sĩ Trịnh Ngọc Trung và tiến sĩ Hoàng Văn Hà đã hướng dẫn tận tình trong suốt quá trình nghiên cứu.
	
	Tôi chân thành cám ơn quý thầy, cô trong khoa Toán-Tin, Trường đại học khoa học tự nhiên thành phố Hồ Chí Minh đã tận tình truyền đạt kiến thức trong những năm tôi học tập ở trường.

	Cuối cùng, tôi xin chúc quý thầy, cô dồi dào sức khỏe và thành công trong sự nghiệp cao quý.
	
\end{acknowledgments}
	
\begin{abstract}
	Nội dung chính của luận văn nhằm tìm hiểu, nghiên cứu xây dựng hệ thống sử dụng các mô hình học máy và điện toán đám mây vào theo dõi quá trình nuôi vi tảo thời gian thực trong công nghiệp dựa trên những công trình, công nghệ mới được nghiên cứu và phát
triển trong những năm gần đây. Trong quá trình nghiên cứu, chúng tôi đã
tiến hành tổng hợp, đánh giá ưu và nhược điểm của cách phương pháp, công nghệ đã và đang
được nghiên cứu, sử dụng. Tiếp cận vấn đề theo nhiều hướng khác nhau, chúng tôi thực hiện một số
phương pháp sử dụng máy học để dự đoán nồng độ chất, thị giác máy tính để nhận diện bọt khí và xây dựng nền tảng theo dõi thời gian thực trong quá trình nuôi vi tảo. Bên
cạnh việc hoàn thành nội dung của đề tài, nhóm chúng tôi đã nghiên cứu thêm một số phần để
từ đó đặt nền móng cho các nghiên cứu sau này. Phần còn lại của luận văn tập trung vào việc
đánh giá mô hình,xây dựng hệ thống và kết quả đạt được, đồng thời phân tích ưu nhược điểm của mô hình và hệ thống thực hiện
và thảo luận những vấn đề mà mô hình và hệ thống còn gặp phải. Cuối cùng, nhóm chúng tôi đề xuất hướng
phát triển tiếp theo của đề tài trong tương lai.
\end{abstract}	


\tableofcontents
%\listofsymbols
% \listoftables
\listoffigures
%\listofalgorithms


\mainmatter

\fancyhead{}  % Clears all page headers and footers
%\rhead{\thepage}  % Sets the right side header to show the page number
%\lhead{}  % Clears the left side page header
%\fancyfoot[positions]{footer}
\renewcommand{\footrulewidth}{0.4pt}

\pagestyle{fancy}  % Finally, use the "fancy" page style to implement the FancyHdr headers
%
\chapter{Giới thiệu tổng quan vấn đề}
	
\section{Giới thiệu đề tài}

\section{Mục tiêu của đề tài}
Mục tiêu của đề tài là nghiên cứu, hiểu và hiện thực một số phương pháp học sâu để phát hiện hướng nhìn của con người qua hình ảnh.

Một số vấn đề đặt ra: 
\begin{itemize}
\item Làm thế nào để giải quyết bài toán trên?
\item Cách tiếp cận như thế nào?
\item Những công nghệ nào đã và hiện đang được sử dụng?
\item Hướng cải tiến?...
\end{itemize}

Như vậy để thực hiện theo đúng mục tiêu của đề tài cần xác định một số công việc phải giải quyết như sau:
\begin{itemize}
\item Tìm kiếm và thu thập dữ liệu phù hợp với nội dung đề tài.
\item Tìm hiểu các phương pháp tiếp cận đã được hiện thực
\item Lựa chọn mô hình phù hợp
\item Lên kế hoạch hiện thực, phát triển hệ thống nhận diện huấn luyện và kiểm thử.
\end{itemize}
\section{Giới thiệu real-time platform}

\section{Cấu trúc luận văn}
Trong giai đoạn luận văn đề tài nhóm đã thực hiện được một số công việc liên quan sẽ trình bày trong báo cáo như sau:

\begin{itemize}
\item Chương 1: Giới thiệu tổng quan vấn đề
\item Chương 2: Xây dựng real-time platform
\item Chương 3: Model Yolo và ứng dụng
\item Chương 4: Kết quả thí nghiệm
\item Chương 5: Tổng kết, đánh giá và định hướng kế hoạch phát triển.
\end{itemize}

\chapter{Xây dựng real-time platfrom trên Azure}

\section{Xây dựng mô hình dự đoán dựa trên tín hiệu điện}
\subsection{Chuẩn bị dữ liệu}
\subsection{Xử lý dữ liệu}
\subsection{Thuật toán Partial Least Squares Regression (PLSR)}
\subsection{Tiến hành huấn luyện và đánh giá mô hình}
\section{Tiến hành xây dựng real-time platform}
\subsection{Thành phần chính}
\subsection{Kiến trúc và nguyên lý hoạt động}
\subsection{Đánh giá chi phí vận hành}
\section{Thử nghiệm và kết luận}

\chapter{Mô hình Yolo và Ứng dụng}
\section{Kiến thức nền tảng}
\section{Mô hình Yolo}

\section{Thuật toán chính}

\section{Xây dựng ứng dụng phát hiện bọt khí trong nuôi vi tảo bằng Yolo}


\chapter{Thực nghiệm và đánh giá}

\section{...}
\section{...}
\section{...}

\chapter{Tổng kết, đánh giá và định hướng kế hoạch phát triển.}

\section{...}
\section{...}
\section{...}

% \printbibliography
% \LaTeX{} 
% \cite{Krizhevsky2012}
% \bibliographystyle{plain} % We choose the "plain" reference style
% \bibliography{refs}% Entries are in the refs.bib file
% \bibliography{refs}{}
% \bibliographystyle{plain}
%-	Danh mục TL tham khảo
%-	Phụ lục (nếu có)


\begin{thebibliography}{99}

% \bibitem{9direction} 
% Chi Zhang, Rui Yao, Jinpeng Cai
% \textit{Efficient Eye Typing with 9 direction Gaze Estimation}.

% \bibitem{appearance} 
% Xucong Zhang, Yusuke Sugano, Mario Fritz, Andreas Bulling
% \textit{Appearance-Based Gaze Estimation in the Wild}. 


\bibitem{plsr} 
Svante Wold, Michael Sjostrom, Lennart Eriksso, PLS-regression: a basic tool of chemometrics, 2001


\bibitem{Yolofly} 
Dillon Reis, Jordan Kupec, Jacqueline Hong, Ahmad Daoudi Georgia Institute of Technology, Real-Time Flying Object Detection with YOLOv8,2023
\\\url{ https://www.semanticscholar.org/reader/231a434f8fac0b01cbc05890b283f4d9da4cb100}



% \bibitem{AReviewandAnalysisofEyeGazeEstimation} 
% Anuradha Kar and Peter M. Corcoran, A Review and Analysis of Eye-Gaze Estimation Systems Algorithms and Performance Evaluation Methods in Consumer Platforms
% \\\url{https://www.semanticscholar.org/paper/A-Review-and-Analysis-of-Eye-Gaze-Estimation-and-in-Kar-Corcoran/ae0a0ee1c6e2adcddffebf9b0e429a25b7d9c0e1}


% \bibitem{tangconv}
% \url{https://developer.apple.com/library/content/documentation/Performance/Conceptual/vImage/ConvolutionOperations/ConvolutionOperations.html}

% \bibitem{lenet5}
% Y. Lecun, L.Boutou, and Y.Bengio, Gradient-based learning applied to document recognition, Proceedings of the IEEE, vol. 88, no. 11, pp. 2278 – 2324, Nov. 1998.

% \bibitem{maxpool}
% Denny Britz,
% \url{http://www.wildml.com/2015/11/understanding-convolutional-neural-networks-for-nlp/}

% \bibitem{cnn}
% Brandon Rohrer, \url{http://brohrer.github.io/how_convolutional_neural_networks_work.html}

% \bibitem{fullconnect}
% Trần Thế Anh, 
% \url{http://labs.septeni-technology.jp/technote/ml-20-convolution-neural-network-part-3/}
% \bibitem{ptha}
% Lương Quốc An, 
% \url{http://nhiethuyettre.net/mang-no-ron-tich-chap-convolutional-neural-network/}

% \bibitem{inception}
% \url{https://leonardoaraujosantos.gitbooks.io/artificial-inteligence/content/googlenet.html}

% \bibitem{softmax}
% Giáo trình Mạng neural, Tác giả: Phan Văn Hiền – Trường Đại học Bách khoa Đà Nẵng, 2013

% \bibitem{alexnet} Aarshay Jain, 
% \url{https://www.analyticsvidhya.com/blog/2016/04/deep-learning-computer-vision-introduction-convolution-neural-networks/}
% \bibitem{dataset}
% \url{https://www.mpi-inf.mpg.de/departments/computer-vision-and-multimodal-computing/research/gaze-based-human-computer-interaction/its-written-all-over-your-face-full-face-appearance-based-gaze-estimation/}

% \bibitem{}
% \url{https://www.tensorflow.org/versions/r0.12/get_started/basic_usage}

% \bibitem{gglenet}
% Christian Szegedy, Wei Liu, Yangqing Jia, Pierre Sermanet, Scott Reed, Dragomir Anguelov, Dumitru Erhan, Vincent Vanhouke, Andrew Rabinovich. \textit{Going deeper with convolutions}

% \bibitem{renset}
% Kaiming He, Xiangyu Zhang, Shaoqing Ren, Jian Sun \textit{Deep Residual Learning for Image Recognition}

% \bibitem{tensor} Trần Thế Anh, 
% \url{http://labs.septeni-technology.jp/technote/ml-18-convolution-neural-network-part-1/}

% \bibitem{mangcnn}
% \url{https://www.kernix.com/blog/a-toy-convolutional-neural-network-for-image-classification-with-keras_p14}
% \bibitem{GazeCaptureEyeTracking}
% Kyle Krafka- Aditya Khosla- Petr Kellnhofer- Harini Kannan- Suchendra Bhandarkar- Wojciech Matusik- Antonio Torralba, Eye Tracking for Everyone
% \url{http://gazecapture.csail.mit.edu/}

% \bibitem{GazeCapturegit}
% Kyle Krafka and Aditya Khosla and Petr Kellnhofer and Harini Kannan and Suchendra Bhandarkar and Wojciech Matusik and Antonio Torralba, Eye Tracking for Everyone Code Dataset and Models
% \url{https://github.com/CSAILVision/GazeCapture}

% \bibitem{eyetrackingapplication}
% \url{https://medium.com/@taolu_99738/developing-of-eye-tracking-application-for-smartphone-b875c50ee0c3}

\end{thebibliography}
\end{document}